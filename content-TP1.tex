{
\graphicspath{{TP1/}}
\newpage
\section{Introduction}
\label{sec:TP1/introduction}
\subfile{TP1/introduction.tex}


%%% Put this secton in an introduction one for the global report %%%
%\section{Ring}
%\label{sec:TP1/ring}
%\subfile{TP1/ring.tex}

\section{Context}\
\label{sec:TP1/context}
\subfile{TP1/context.tex}

\section{Statistical Analysis}\
\label{sec:TP1/static}
\subfile{TP1/static.tex}

\section{Dynamic Analysis}\
\label{sec:TP1/dynamic}
\subfile{TP1/dynamic.tex}

\section{Ring architecture}
\begin{figure}[h!]
\centering
\includegraphics[width=120mm]{images/architecture.png}
\caption{Ring Components}
\end{figure}
\begin{figure}[h!]

\centering
\includegraphics[width=120mm]{images/4+1.png}
\caption{4+1 architectural view model \cite{modeling}}
\label{fig:4+1 arch}
\end{figure}

\section{4+1 Model}
The "4+1" model is a describing model of systems called software-intensives. The model is based on concurrent views that are able to describe in general \cite{blueprints}. Figure \ref{fig:4+1 arch} illustrates the 4+1 model.
The model is composed of 4 Views: Process View, Physical, Logical and Deployment. But also, it includes some scenarios. The 4+1 Ring's Model is presented below.

\subsection{Development view}
\label{sec:TP1/development}
\subfile{TP1/development.tex}

\subsection{Logical view}
\label{sec:TP1/logical}
\subfile{TP1/logical.tex}

\subsection{Process view}
\label{sec:TP1/process}
\subfile{TP1/process.tex}


\subsection{Physical view}
\label{sec:TP1/physical}
\subfile{TP1/physical.tex}

\subsection{Scenarios}
\label{sec:TP1/scenarios}
\subfile{TP1/scenarios.tex}

\subsection{Performance View}
\label{sec:performance}
\subfile{TP1/performance.tex}

\section{Design Patterns}
\label{sec:patterns}
\subfile{TP1/patterns.tex}

\section{Conclusion}
\label{sec:conclusion}
\subfile{TP1/conclusion.tex}

}