\documentclass[11pt]{report}
\usepackage[utf8]{inputenc}
\usepackage{graphicx}
\usepackage{amssymb}
\usepackage[english]{babel}
\usepackage{fullpage}
\usepackage{ragged2e}
\usepackage{appendix}
\usepackage{subcaption}
\usepackage{indentfirst}
\usepackage{attachfile}

\usepackage{subfiles}
\usepackage[backend=biber]{biblatex}
\addbibresource{TP1/sample.bib}
\addbibresource{TP2/sample.bib}
\addbibresource{TP3/sample.bib}

\def\titre{}
\def\auteur{}
\def\courriel{}
\makeatletter

\title{LOG8430\\Architecture logicielle et conception avanc\'{e}e}
\author{
    Yann-Gaël Guéhéneuc \\
    Fabio Petrillo \\
    D\'{e}partement G\'{e}nie Informatique et G\'{e}nie Logiciel \\
    \'{E}cole Polytechnique de Montr\'{e}al, Qu\'{e}bec, Canada \\
    \texttt{yann-gael.gueheneuc[at]polymtl.ca} \\
    \texttt{Fabio[at]petrillo.com}
}

\date{\today}
\usepackage{url}
\usepackage{float}
\begin{document}
{\let\newpage\relax\maketitle}
\begin{figure}[ht!]
\centering
\includegraphics[width=90mm]{TP1/images/ring.png}
\end{figure}

\paragraph{Students names:}
\centering
\paragraph{} \auteur
Isnaldo Francisco de Melo jr\\
Franck Brazier\\
Stephane Fagnon\\

\centering

\paragraph{Date of the submission:}
\today
\newpage

\tableofcontents
\newpage

\justify

\chapter*{Introduction}
ezf

{\let\clearpage\relax \chapter*{Ring}
\subfile{TP1/ring.tex}
Ring is involved in the Free Software philosophy; that is proved  by the fact that it is a GNU\cite{GNU} package \cite{official_blog}. 
}

\chapter{TP1 - Étude et analyse du Ring }
{
\graphicspath{{TP1/}}
\newpage
\section{Introduction}
\label{sec:TP1/introduction}
\subfile{TP1/introduction.tex}


%%% Put this secton in an introduction one for the global report %%%
%\section{Ring}
%\label{sec:TP1/ring}
%\subfile{TP1/ring.tex}

\section{Context}\
\label{sec:TP1/context}
\subfile{TP1/context.tex}

\section{Statistical Analysis}\
\label{sec:TP1/static}
\subfile{TP1/static.tex}

\section{Dynamic Analysis}\
\label{sec:TP1/dynamic}
\subfile{TP1/dynamic.tex}

\section{Ring architecture}
\begin{figure}[h!]
\centering
\includegraphics[width=120mm]{images/architecture.png}
\caption{Ring Components}
\end{figure}
\begin{figure}[h!]

\centering
\includegraphics[width=120mm]{images/4+1.png}
\caption{4+1 architectural view model \cite{modeling}}
\label{fig:4+1 arch}
\end{figure}

\section{4+1 Model}
The "4+1" model is a describing model of systems called software-intensives. The model is based on concurrent views that are able to describe in general \cite{blueprints}. Figure \ref{fig:4+1 arch} illustrates the 4+1 model.
The model is composed of 4 Views: Process View, Physical, Logical and Deployment. But also, it includes some scenarios. The 4+1 Ring's Model is presented below.

\subsection{Development view}
\label{sec:TP1/development}
\subfile{TP1/development.tex}

\subsection{Logical view}
\label{sec:TP1/logical}
\subfile{TP1/logical.tex}

\subsection{Process view}
\label{sec:TP1/process}
\subfile{TP1/process.tex}


\subsection{Physical view}
\label{sec:TP1/physical}
\subfile{TP1/physical.tex}

\subsection{Scenarios}
\label{sec:TP1/scenarios}
\subfile{TP1/scenarios.tex}

\subsection{Performance View}
\label{sec:performance}
\subfile{TP1/performance.tex}

\section{Design Patterns}
\label{sec:patterns}
\subfile{TP1/patterns.tex}

\section{Conclusion}
\label{sec:conclusion}
\subfile{TP1/conclusion.tex}

}
\newpage
\chapter{TP2 - Réusinage Architectural du code Ring}
{
\graphicspath{{TP2/}}
\newpage

\section{Introduction}
\label{sec:TP2/introduction}
\subfile{TP2/introduction.tex}

\section{Ring}
\label{sec:ring}
\subfile{TP2/ring.tex}

\section{Code Quality Analysis}
\label{sec:TP2/context}
\subfile{TP2/quality.tex}

\section{Ptidej}
\label{sec:TP2/ptidej}
\subfile{TP2/ptidej.tex}

\section{Design Patterns}
\label{sec:TP2/patterns}
\subfile{TP2/patterns.tex}

\section{Anti Pattern}
\label{sec:TP2/antipatterns}
\subfile{TP2/antipatterns.tex}

\section{Problem Exposed}
\label{sec:TP2/p_exposed}
\subfile{TP2/descriptionX.tex}


\newpage
\section{Conclusion}
\label{sec:TP2/conclusion}
\subfile{TP2/conclusion.tex}
}
\newpage
\chapter{TP3 - Implémentation et contribution au projet Ring }
{
\graphicspath{{TP3/}}
\newpage
\section{Introduction}
\label{sec:TP3/introduction}
\subfile{TP3/introduction.tex}


%%%%% This section comes from the  TP2 %%%%%%
%%%%% Update TP2 %%%%%%%%
\section{Description of the problem}
\label{sec:TP3/problem}
\subfile{TP3/description.tex}

\section{Analysis}
\label{sec:TP3/analysis}
\subfile{TP3/analysis.tex}


\section{Pull Request}
\label{sec:TP3/pull-request}
\subfile{TP3/pull-request.tex}


\section{Conclusion}
\label{sec:TP3/conclusion}
\subfile{TP3/conclusion.tex}	


\chapter*{Conclusion}
fzen


\appendix
\chapter{TPs Changes}
%\section{Changes}
\input{TP3/changes.tex}
\chapter{TP1 - Appendices}
\input{TP1/appendix.tex}
\chapter{TP2 - Appendices}
\input{TP2/appendix.tex}
\chapter{TP3 - Appendices}
\input{TP3/appendix.tex}


\printbibliography



\end{document}